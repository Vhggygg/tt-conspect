\subsection{\texorpdfstring{Ромбовидное свойство и параллельная редукция}{Diamond property and parallel reduction}}

\begin{definition}[Ромбовидное свойство (diamond)]
    $G$ обладает ромбовидным свойством, если какие бы ни были $a$, $b$, $c$, что $aGb$, $aGc$, $b \ne c$, найдётся такое $d$, что $bGd$ и $cGd$.
\end{definition}

\begin{example}
    $(<)$ на натуральных числах обладает ромбовидным свойством.
    $(>)$ на натуральных числах не обладает ромбовидным свойством.
\end{example}

$\beta$-редукция не обладает ромбовидным свойством.
\begin{example} % TODO запилить картинку
    \begin{gather*}
        a = (\lambda x . x x)(Ia) \\
        a \rightarrow_{\beta} (Ia)(Ia) = b\\
        a \rightarrow_{\beta} (\lambda x . x x) a = c \\
        b \rightarrow_{\beta} (Ia)a \rightarrow_{\beta} aa \\
        b \rightarrow_{\beta} a(Ia) \rightarrow_{\beta} aa \\
        c \rightarrow_{\beta} aa
    \end{gather*}
    Нет $d$, что $b \rightarrow_{\beta} d$ и $c \rightarrow_{\beta} d$.
\end{example}

\begin{theorem}[Чёрча-Россера] \label{church-rosser}
    $\beta$-редуцируемость обладает ромбовидным свойством.
\end{theorem}

\begin{lemma}
    Если $R$ обладает ромбовидным свойством, то $R^{*}$ обладает ромбовидным свойством.
\end{lemma}

\begin{proof} (Упражнение) \todo % TODO
    \begin{enumerate}
        \item $M_{1}RN_{1}$ и $M_{1}RM_{2}...M_{n-1}RM_{n}$ $\Rightarrow$ есть $N_{2}...N_{n}$: \\
            $N_{1}RN_{2}...N_{n-1}RN_{n}$ и $M_{n}RN_{n}$.
        \item Покажем ромбовидное свойство.
        \qedhere
    \end{enumerate}
\end{proof}

\begin{definition}[Параллельная $\beta$-редукция]
    $A \rightrightarrows_{\beta} B$
    \begin{enumerate}
        \item $A =_{\beta} B$, то $A \rightrightarrows_{\beta}B$
        \item $A \rightrightarrows_{\beta} B$, то $\lambda x.A \rightrightarrows_{\beta} \lambda x . B$
        \item $P \rightrightarrows_{\beta} Q$ и $R \rightrightarrows_{\beta} S$, то $PR \rightrightarrows_{\beta} QS$
        \item $(\lambda x . P) Q \rightrightarrows_{\beta} R_{[x\coloneqq{}S]}$, если 
            $P \rightrightarrows_{\beta}R$ и $Q \rightrightarrows_{\beta} S$.
    \end{enumerate}
\end{definition}

\begin{statement} \label{st-star}
    $(\rightrightarrows_{\beta})$ обладает ромбовидным свойством.
\end{statement}

\begin{proof}
    (Упражнение) \todo % TODO
\end{proof}

\begin{statement} \label{st-A}
    Если $A \rightarrow_{\beta} B$, то $A \rightrightarrows_{\beta} B$.
\end{statement}

\begin{statement} \label{st-B}
    Если $A \rightrightarrows_{\beta} B$, то $A \twoheadrightarrow_{\beta} B$.
\end{statement}

\begin{proof}
    (Упражнение) \todo % TODO
\end{proof}

При этом, обратное не всегда верно.

\begin{example}
    \begin{gather*}
        (\lambda x . x x) (\lambda x . x x x) \twoheadrightarrow_{\beta} (\lambda x . x x x)(\lambda x . x x x)(\lambda x . x x x) \\
        (\lambda x . x x) (\lambda x . x x x) \cancel{\rightrightarrows_{\beta}} (\lambda x . x x x)(\lambda x . x x x)(\lambda x . x x x)
    \end{gather*}
\end{example}

\begin{statement} \label{st-C}
    Из \ref{st-A} и \ref{st-B} следует, что $(\rightarrow_{\beta})^{*} = (\rightrightarrows_{\beta})^{*}$.
\end{statement}

\begin{proof}
    Теорема \nameref{church-rosser} следует из \ref{st-star} и \ref{st-C}.
\end{proof}

\begin{corollary}
    Нормальная форма для $\lambda$-выражения единственна, если существует.
\end{corollary}

\begin{theorem}[Тезис Чёрча]
    Если функция вычислима с помощью механического аппарата, то она вычислима с помощью $\lambda$-выражения.
\end{theorem}

\subsection{\texorpdfstring{Порядок редукции}{Order of reduction}}
\epigraph{"<Завтра! Завтра! Не сегодня!"> "--- так ленивцы говорят.}{Das deutsches Sprichwort}

\begin{definition}
    \begin{align*}
        K &= \lambda x \lambda y . x \\
        I &= \lambda x . x \\
        S &= \lambda x y z . x z (y z)
    \end{align*}
\end{definition}
$I$ выражается через $S$ и $K$: $I = S K K$.

\begin{statement} \label{SK-basis}
    Пусть $A$ "--- замкнутое $\lambda$-выражение. Тогда найдётся выражение $T$, состоящее только из $S$ и $K$, что $A =_{\beta}T$.
\end{statement}

\begin{example}
    тут какой-то пример с омегой, подскажите чё там было, \todo % TODO
\end{example}

\begin{definition}[Нормальный порядок редукции]
    Нормальным порядком редукции называется редукция самого левого $\beta$-редекса.
\end{definition}
"<Ленивые вычисления"> (ну, почти, в ленивых ещё есть меморизация)

\begin{definition}[Аппликативный порядок редукции]
    Самый левый из самых вложенных.
\end{definition}
"<Энергичные вычисления">

\begin{statement}
    Если нормальная форма существует, она может быть достигнута нормальным порядком редукции.
\end{statement}

\subsection{\texorpdfstring{Парадокс Карри}{Curry's paradox}}

\epigraph{Если это утверждение верно, то русалки существуют.}

Попробуем построить логику на основе $\lambda$-исчисления.
Введём комбнатор-импликацию, обозначим $(\supset)$. Введём M.P. и правила:
\begin{enumerate}
    \item $A \supset A$
    \item $(A \supset (A \supset B)) \supset (A \supset B)$
    \item $A =_{\beta} B$, тогда $A \supset B$
\end{enumerate}

Покажем, как в полученной логике можно доказать любое утверждение.
Введём обозначение: $Y_{\supset a} \equiv Y (\lambda t . t \supset a) =_{\beta} Y (\lambda t . t \supset a) \supset a$.

\begin{tabular}{lll}
    1) & $Y_{\supset a} \supset Y_{\supset a}$ & (схема аксиом) \\
    2) & $Y_{\supset a} \supset (Y_{\supset a} \supset a)$ & (можно доказать) \\
    3) & $(Y_{\supset a} \supset Y_{\supset a} \supset a) \supset (Y_{\supset a} \supset a)$ & (схема аксиом) \\
    4) & $Y_{\supset a} \supset a$ & (M.P.) \\
    5) & $(Y_{\supset a} \supset a) \supset Y_{\supset a}$ & (третье правило) \\
    6) & $Y_{\supset a}$ & (M.P.) \\
    7) & $a$ & (M.P.)
\end{tabular}

Получается, что данная логика противоречива.

\subsection{\texorpdfstring{Импликационный фрагмент ИИВ}{Implication fragment of intuitionistic logic}}

\begin{definition}[импликационный фрагмент ИИВ]
    Рассмотрим интуиционистское исчисление высказываний.
    \begin{enumerate}
        \item Введём схему аксиом:
        \[
            \infer{\Gamma, \varphi \vdash \varphi}{}
        \]
        \item Правило введения импликации:
        \[
            \infer{\Gamma \vdash \varphi \rightarrow \psi}{\Gamma, \varphi \vdash \psi}
        \]
        \item И правило удаления импликации:
        \[
            \infer{\Gamma \vdash \psi}{\Gamma \vdash \varphi \rightarrow \psi & \Gamma \vdash \varphi}
        \]
    \end{enumerate}

    Мы построили импликационный фрагмент ИИВ (и.ф.и.и.в).
\end{definition}

\begin{example} Докажем $\varphi \rightarrow \psi \rightarrow \varphi$:
\[
    \infer[(2)]
        { \vdash \varphi \rightarrow (\psi \rightarrow \varphi) }
        { \infer[(2)]
            { \varphi \vdash \psi \rightarrow \varphi }
            { \infer[(1)]
                { \varphi, \psi \vdash \varphi}
                {}
            }
        }
\]
\end{example}

\begin{theorem}
    И.ф.и.и.в полон в моделях Крипке, то есть $\Gamma \vdash \varphi$ т.и.т.т.,
    когда для любой модели крипке $C$ из $\Vdash_C \Gamma$ следует $\Vdash_C \varphi$.
\end{theorem}

\begin{proof}
    Рассмотрим модель Крипке вида $W = \left\{\Delta \mid \Gamma \subseteq \Delta, \Delta\text{ замкнуто относительно }\vdash\right\}$,
    $\Gamma \leq \Delta$ если $\Gamma \subseteq \Delta$.
    Индукцией по структуре $\varphi$ покажем, что $\Delta \Vdash \varphi$ т.и.т.т., когда $\Delta \vdash \varphi$:
    \begin{enumerate}
        \item Пусть $\varphi \equiv x$ "--- переменная. Тогда $\Gamma \vdash \varphi$ эквивалентно $x \in \Gamma$, что эквивалентно $\Vdash x$ (по определению).
        \item Пусть $\varphi \equiv \alpha \rightarrow \beta$.
        \begin{enumerate}[label=(\asbuk*)]
            \item Пусть $\Delta \vdash \varphi$.
            Рассмотрим такое $\Delta'$, что $\Delta \leq \Delta'$ и $\Delta' \Vdash \alpha$.
        \end{enumerate}
        ой всё \todo
    \end{enumerate}
\end{proof}

\begin{corollary}
    И.ф.и.и.в замкнут относительно выводимости.
\end{corollary}
Если некоторое утверждение выводится в ИИВ ($\vdash_{и} \varphi$) и содержит только импликации,
то оно выводится и в и.ф.и.и.в. ($\vdash_{и \rightarrow} \varphi$).
