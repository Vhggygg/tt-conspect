\documentclass{article}

\usepackage{amsmath}
\usepackage{amssymb}
\usepackage{amsthm}
\usepackage{mathtext}
\usepackage{mathtools}
\usepackage[T1,T2A]{fontenc}
\usepackage[utf8]{inputenc}
%\usepackage{geometry}
\usepackage[left=2cm,right=2cm,top=2cm,bottom=2cm,bindingoffset=0cm]{geometry}
\usepackage[mathscr]{euscript}
\usepackage{microtype}
\usepackage{bnf}
\usepackage{enumitem}
\usepackage{bm}
\usepackage{listings}
\usepackage{cancel}
\usepackage{proof}
\usepackage{epigraph}
\usepackage{titlesec}
\usepackage{xcolor}
\usepackage{stmaryrd}
\usepackage{cellspace}
\usepackage{cmll}
\usepackage{tabularx}
\usepackage{verbatimbox}
\usepackage{multirow}

\usepackage[hidelinks]{hyperref}

%\setmainfont[Ligatures=TeX,SmallCapsFont={Times New Roman}]{Palatino Linotype}

\usepackage[russian]{babel}
\selectlanguage{russian}

\hypersetup{%
    colorlinks=true,
    linkcolor=blue
}

\AddEnumerateCounter{\Asbuk}{\@Asbuk}{\CYRM}
\AddEnumerateCounter{\asbuk}{\@asbuk}{\cyrm}

%http://tex.stackexchange.com/questions/31526/macro-for-left-and-right/58641#58641

\DeclareMathOperator{\rank}{rk}
\DeclareMathOperator{\inj}{inj}
\DeclareMathOperator{\FV}{FV}

\title{Теория типов}
\author{$\heartsuit$}
%Человек, который три семестра собирался верстать конспекты \\
%Человек, который иногда вспоминает, что в ХПИ не так уж и плохо \\
%Человек, который всегда рад помочь $\heartsuit$}
\date{}

\begin{document}

\theoremstyle{definition}
\newtheorem*{definition}{Определение}
\theoremstyle{plain}
\newtheorem{theorem}{Теорема}[section]
\newtheorem{axiom}{Аксиома}
\newtheorem{lemma}[theorem]{Лемма}
\newtheorem{statement}[theorem]{Утверждение}
\newtheorem{corollary}[theorem]{Следствие}
\theoremstyle{remark}
\newtheorem*{example}{Пример}
\newtheorem{property}[theorem]{Свойство}

\lstset{language=C++}

\newcommand{\todo}{\textsc{\textbf{TODO}}\ }
\newcommand{\abs}[1]{\left|#1\right|}
\newcommand{\set}[1]{\left\{#1\right\}}
\newcommand{\xx}{ч}
\newcommand{\rr}{к}
\newcommand{\case}[3]{\text{\textbf{case}}~#1~#2~#3}
\newcommand{\pack}[2]{\text{\textbf{pack}}~#1~\text{\textbf{to}}~#2}
\newcommand{\abstype}[4]{\text{\textbf{abstype}}~#1~\text{\textbf{with}}~#2~\text{\textbf{in}}~#3~\text{\textbf{is}}~#4}
\newcommand{\lett}[2]{\text{\textbf{let}}~#1~\text{\textbf{in}}~#2}

\maketitle
\tableofcontents
%\newpage
\newcommand{\sectionbreak}{\clearpage}

%\renewcommand{\arraystretch}{1.5}
\setlength\cellspacetoplimit{5pt}
\setlength\cellspacebottomlimit{5pt}
\setlength{\inferLineSkip}{4pt}

\section{\texorpdfstring{$\lambda$-исчисление}{Lambda calculus}}

\subsection{\texorpdfstring{Введение}{Introduction}}
\epigraph{Смысла в этом нет.}{Д.Г.}

\begin{definition}[$\lambda$-выражение]
    $\lambda$-выражение "--- выражение, удовлетворяющее грамматике:
    \begin{bnf}
    \begin{alignat*}{3}
        \Lambda ::= & \lambda{}x.\Lambda{} \qquad && (абстракция) \\
                  | & \Lambda{}\Lambda{}          && (аппликация) \\
                  | & x                                           \\
                  | & \left(\Lambda\right)
    \end{alignat*}
    \end{bnf}
    \begin{enumerate}[label=(\asbuk*)]
        \item аппликация левоассоциативна
        \item абстракция распространяется как можно дальше вправо
    \end{enumerate}
\end{definition}

\begin{example}
    $((\lambda{} z.(z (y z))) (z x) z) = (\lambda{} z.z (y z)) (z x) z$
\end{example}
%[(а)]
%[(б)]
%[(в)]

Есть понятия связанного и свободного вхождения переменной (аналогично ИП).
$\lambda{}x.A$ связывает все свободные вхождения $x$ в $A$.
Договоримся, что:
\begin{enumerate}[label=(\asbuk*)]
    \item Переменные "--- $x$, $a$, $b$, $c$.
    \item Термы (части $\lambda$-выражения) "--- $X$, $A$, $B$, $C$.
    \item Фиксированные переменные обозначаются буквами из начала алфавита, метапеременные "--- из конца.
\end{enumerate}

\begin{definition}[$\alpha$-эквивалентность]
    $A$ и $B$ называются $\alpha$-эквивалентными ($A=_{\alpha}B$), если выполнено одно из следующих условий:
    \begin{enumerate}
        \item $A\equiv{}x$ и $B\equiv{}x$.
        \item $A\equiv{}\lambda{}x.P$, $B\equiv{}\lambda{}y.Q$ и $P_{[x\coloneqq{}t]}=_{\alpha}Q_{[y\coloneqq{}t]}$, где $t$ "--- новая переменная.
        \item $A\equiv{}PQ$, $B\equiv{}RS$ и $P=_{\alpha}R$, $Q=_{\alpha}S$.
    \end{enumerate}
\end{definition}

\begin{example}
    $\lambda{}x.\lambda{}y.xy=_{\alpha}\lambda{}y.\lambda{}x.yx$.
    \begin{proof}
        \begin{alignat*}{2}
            \lambda{}y.ty=_{\alpha}\lambda{}x.tx &\implies \lambda{}x.\lambda{}y.xy=_{\alpha}\lambda{}y.\lambda{}x.yx \\
            tz=_{\alpha}tz &\implies \lambda{}y.ty=_{\alpha}\lambda{}x.tx
        \end{alignat*}
        $tz=_{\alpha}tz$ верно по третьему условию.
    \end{proof}
\end{example}

\begin{definition}[$\beta$-редекс]
    Терм вида $\left(\lambda{}a.A\right)B$ называется $\beta$-редексом.
\end{definition}

\begin{example}
    В выражении
    $
        (
            \lambda{}f.
                \underset{A_2}{\underline{
                    (\lambda{}x.\overset{A_1}{\overline{f(xx)}})
                    \overset{B_1}{\overline{(\lambda{}x.f(xx))}}
                }}
        )\underset{B_2}{\underline{g}}
    $ два $\beta$-редекса.
\end{example}

\begin{definition}
    Множество $\lambda$-термов $\bm{\Lambda}$ назовём множеством классов эквивалентности $\Lambda$ по $(=_{\alpha})$.
\end{definition}

\begin{definition}[$\beta$-редукция]
    $A\rightarrow_{\beta}B$ (состоят в отношении $\beta$-редукции), если выполняется одно из условий:
    \begin{enumerate}
        \item $A\equiv{}PQ$, $B\equiv{}RS$ и
        \begin{alignat*}{3}
            &\text{либо } P\rightarrow_{\beta}R  &&\text{ и } Q=_{\alpha}S \\
            &\text{либо } P=_{\alpha}R           &&\text{ и } Q\rightarrow_{\beta}S
        \end{alignat*}
        \item $A\equiv{}\lambda{}x.P$, $B\equiv{}\lambda{}x.Q$, $P\rightarrow_{\beta}Q$ ($x$ из какого-то класса из $\bm{\Lambda}$).
        \item $A\equiv{}(\lambda{}x.P)Q$, $B\equiv{}P_{[x\coloneqq{}Q]}$, $Q$ свободно для подстановки в $P$ вместо $x$.
    \end{enumerate}
\end{definition}

\subsection{\texorpdfstring{Числа Чёрча}{Church numerals}}
\epigraph{Хотите знать, что такое истина?}{Д.Г.}
%\paragraph{Итак, лулзы.} Хотите знать, что такое истина?

\newcommand{\T}{\mathrm{T}}
\newcommand{\F}{\mathrm{F}}
\newcommand{\Not}{\mathrm{Not}}
\begin{alignat*}{2}
    \T   &= \lambda{}x\lambda{}y.x \\
    \F   &= \lambda{}x\lambda{}y.y \\
    \Not &= \lambda{}a.a\F\T
\end{alignat*}

Похоже на тип boolean, не правда ли?
\begin{example}
    \[
        \Not\ \T = (\lambda{}a.a\F\T)\T \rightarrow_{\beta}
            \T\F\T = (\lambda{}x.\lambda{}y.x)\F\T \rightarrow_{\beta}
            (\lambda{}y.\F)\T \rightarrow_{\beta}
            \F
    \]
\end{example}

Можно продолжить:
\begin{alignat*}{2}
    \mathrm{And} &= \lambda{}a.\lambda{}b.ab\mathrm{F} \\
    \mathrm{Or}  &= \lambda{}a.\lambda{}b.a\mathrm{T}b
\end{alignat*}

Попробуем определить числа:
\begin{definition}[Чёрчевский нумерал]
\[
    \overline{n}=\lambda{}f.\lambda{}x.f^{n}x \text{,\quadгде\quad}
    f^{n}x=
    \begin{cases}
        f\left(f^{n-1}x\right) &, n > 0 \\
        x                      &, n = 0
    \end{cases}
\]
\end{definition}

\begin{example}
\[
    \overline{3} = \lambda f . \lambda x . f (f (f x))
\]
\end{example}

Несложно определить прибавление единицы к такому нумералу:
\[
    (+1) = \lambda{}n.\lambda{}f.\lambda{}x.f(nfx) \\
\]
\begin{example}
    \[
        (+1) \overline{1} =
        (\lambda n . \lambda f . \lambda x . f (n f x)) (\lambda f . \lambda x . f x) \rightarrow_{\beta}
        \lambda f . \lambda x . f ((\lambda f . \lambda x . f x) f x) \twoheadrightarrow_{\beta}
        \lambda f . \lambda x . f (f x) =
        \overline{2}
    \]
\end{example}

\begin{definition}[$\eta{}$-эквивалентность]
    \[
        \lambda x . f x =_{\eta} f
    \]
\end{definition}
Аналог из C++: если \lstinline$int f(int x)$, то результат её вычисления равен результату вычисления\\
    \lstinline$[ ] (int x) { return f(x); }$ .

Арифметические операции:
\begin{alignat*}{2}
    \mathrm{IsZero} &= \lambda{}n.n(\lambda{}x.\F)\T \\
    \mathrm{IsEven} &= \lambda{}n.n\ \Not\ \T \\
    \mathrm{Add} &= \lambda{}a.\lambda{}b.\lambda{}f.\lambda{}x.a f (b f x) \\
    \mathrm{Mul} &= \lambda{}a.\lambda{}b.a (\mathrm{Add}\ b) \overline{0} \\
    \mathrm{Pow} &= \lambda{}a.\lambda{}b.b (\mathrm{Mul}\ a) \overline{1} \\
    \mathrm{Pow}^{*} &= \lambda{}a.\lambda{}b.b a
\end{alignat*}

Для того, чтобы определить $(-1)$, сначала определим "пару":
\begin{alignat*}{2}
    \left<a,b\right> &= \lambda f.f a b \\
    \mathrm{First} &= \lambda p . p \T \\
    \mathrm{Second} &= \lambda p . p \F
\end{alignat*}

$n$ раз применим функцию $f\left(\left<a,b\right>\right) = \left<b,b+1\right>$ и возьмём первый элемент пары:
\[
    (-1) = \lambda n . \mathrm{First} \left(n\ (\lambda p . \left<\left(\mathrm{Second}\ p\right), (+1)\ (\mathrm{Second}\ p)\right>)\
    \left<\overline{0},\overline{0}\right>\right)
\]

Введём сокращение записи:
\[
    \lambda x y . A = \lambda x . \lambda y . A
\]

\begin{definition}[Нормальная форма] \mbox{} \\
    Терм $A$ "--- нормальная форма (н.ф.), если в нём нет $\beta$-редексов. \\
    Нормальной формой $A$ называется такой $B$, что $A \twoheadrightarrow_{\beta} B$, $B$ "--- н.ф. \\
    $\twoheadrightarrow_{\beta}$ "--- транзитивно-рефлексивное замыкание $\rightarrow_{\beta}$.
\end{definition}

\begin{statement}
    Существует $\lambda$-выражение, не имеющее н.ф.
\end{statement}

\begin{definition}[Комбинатор]
    Комбинатор "--- $\lambda$-выражение без свободных переменных.
\end{definition}

\begin{definition}
    \begin{gather*}
        \Omega = \omega \omega \\
        \omega = \lambda x . x x
    \end{gather*}
\end{definition}

$\Omega$ не имеет нормальной формы.

\begin{definition}[Комбинатор неподвижной точки]
    \[
        Y = \lambda f . (\lambda x . f (x x)) (\lambda x . f (x x))
    \]
\end{definition}

\begin{definition}[$\beta$-эквивалентность]
    $A=_{\beta}B$, если $\exists C : C \twoheadrightarrow_{\beta} A, C \twoheadrightarrow_{\beta}B$
\end{definition}

\begin{statement}
    \[
        Yf =_{\beta} f(Yf)
    \]
\end{statement}

\begin{proof} (на лекции не давалось)
    \begin{align*}
        Yf &=_{\beta} (\lambda f . (\lambda x . f (x x)) (\lambda x . f (x x))) f \\
           &=_{\beta} (\lambda x . f (x x)) (\lambda x . f (x x)) \\
           &=_{\beta} f ((\lambda x . f (x x)) (\lambda x . f (x x))) \\
           &=_{\beta} f (Y f)
    \qedhere
    \end{align*}
\end{proof}

Таким образом, с помощью $Y$-комбинатора можно определять рекурсивные функции.
\begin{example}
    \[
        \mathrm{Fact} = Y (\lambda{} f n . \mathrm{IsZero}\ n\ \overline{1}\ (\mathrm{Mul}\ n\ (f\ (-1)\ n)))
    \]
\end{example}

\todo % пояснение

\subsection{\texorpdfstring{Ромбовидное свойство и параллельная редукция}{Diamond property and parallel reduction}}

\begin{definition}[Ромбовидное свойство (diamond)]
    $G$ обладает ромбовидным свойством, если какие бы ни были $a$, $b$, $c$, что $aGb$, $aGc$, $b \ne c$, найдётся такое $d$, что $bGd$ и $cGd$.
\end{definition}

\begin{example}
    $(<)$ на натуральных числах обладает ромбовидным свойством.
    $(>)$ на натуральных числах не обладает ромбовидным свойством.
\end{example}

$\beta$-редукция не обладает ромбовидным свойством.
\begin{example} % TODO запилить картинку
    \begin{gather*}
        a = (\lambda x . x x)(Ia) \\
        a \rightarrow_{\beta} (Ia)(Ia) = b\\
        a \rightarrow_{\beta} (\lambda x . x x) a = c \\
        b \rightarrow_{\beta} (Ia)a \rightarrow_{\beta} aa \\
        b \rightarrow_{\beta} a(Ia) \rightarrow_{\beta} aa \\
        c \rightarrow_{\beta} aa
    \end{gather*}
    Нет $d$, что $b \rightarrow_{\beta} d$ и $c \rightarrow_{\beta} d$.
\end{example}

\begin{theorem}[Чёрча-Россера] \label{church-rosser}
    $\beta$-редуцируемость обладает ромбовидным свойством.
\end{theorem}

\begin{lemma}
    Если $R$ обладает ромбовидным свойством, то $R^{*}$ обладает ромбовидным свойством.
\end{lemma}

\begin{proof} (Упражнение) \todo % TODO
    \begin{enumerate}
        \item $M_{1}RN_{1}$ и $M_{1}RM_{2}...M_{n-1}RM_{n}$ $\Rightarrow$ есть $N_{2}...N_{n}$: \\
            $N_{1}RN_{2}...N_{n-1}RN_{n}$ и $M_{n}RN_{n}$.
        \item Покажем ромбовидное свойство.
        \qedhere
    \end{enumerate}
\end{proof}

\begin{definition}[Параллельная $\beta$-редукция]
    $A \rightrightarrows_{\beta} B$
    \begin{enumerate}
        \item $A =_{\beta} B$, то $A \rightrightarrows_{\beta}B$
        \item $A \rightrightarrows_{\beta} B$, то $\lambda x.A \rightrightarrows_{\beta} \lambda x . B$
        \item $P \rightrightarrows_{\beta} Q$ и $R \rightrightarrows_{\beta} S$, то $PR \rightrightarrows_{\beta} QS$
        \item $(\lambda x . P) Q \rightrightarrows_{\beta} R_{[x\coloneqq{}S]}$, если 
            $P \rightrightarrows_{\beta}R$ и $Q \rightrightarrows_{\beta} S$.
    \end{enumerate}
\end{definition}

\begin{statement} \label{st-star}
    $(\rightrightarrows_{\beta})$ обладает ромбовидным свойством.
\end{statement}

\begin{proof}
    (Упражнение) \todo % TODO
\end{proof}

\begin{statement} \label{st-A}
    Если $A \rightarrow_{\beta} B$, то $A \rightrightarrows_{\beta} B$.
\end{statement}

\begin{statement} \label{st-B}
    Если $A \rightrightarrows_{\beta} B$, то $A \twoheadrightarrow_{\beta} B$.
\end{statement}

\begin{proof}
    (Упражнение) \todo % TODO
\end{proof}

При этом, обратное не всегда верно.

\begin{example}
    \begin{gather*}
        (\lambda x . x x) (\lambda x . x x x) \twoheadrightarrow_{\beta} (\lambda x . x x x)(\lambda x . x x x)(\lambda x . x x x) \\
        (\lambda x . x x) (\lambda x . x x x) \cancel{\rightrightarrows_{\beta}} (\lambda x . x x x)(\lambda x . x x x)(\lambda x . x x x)
    \end{gather*}
\end{example}

\begin{statement} \label{st-C}
    Из \ref{st-A} и \ref{st-B} следует, что $(\rightarrow_{\beta})^{*} = (\rightrightarrows_{\beta})^{*}$.
\end{statement}

\begin{proof}
    Теорема \nameref{church-rosser} следует из \ref{st-star} и \ref{st-C}.
\end{proof}

\begin{corollary}
    Нормальная форма для $\lambda$-выражения единственна, если существует.
\end{corollary}

\begin{theorem}[Тезис Чёрча]
    Если функция вычислима с помощью механического аппарата, то она вычислима с помощью $\lambda$-выражения.
\end{theorem}

\subsection{\texorpdfstring{Порядок редукции}{Order of reduction}}
\epigraph{"<Завтра! Завтра! Не сегодня!"> "--- так ленивцы говорят.}{Das deutsches Sprichwort}

\begin{definition}
    \begin{align*}
        K &= \lambda x \lambda y . x \\
        I &= \lambda x . x \\
        S &= \lambda x y z . x z (y z)
    \end{align*}
\end{definition}
$I$ выражается через $S$ и $K$: $I = S K K$.

\begin{statement} \label{SK-basis}
    Пусть $A$ "--- замкнутое $\lambda$-выражение. Тогда найдётся выражение $T$, состоящее только из $S$ и $K$, что $A =_{\beta}T$.
\end{statement}

\begin{example}
    тут какой-то пример с омегой, подскажите чё там было, \todo % TODO
\end{example}

\begin{definition}[Нормальный порядок редукции]
    Нормальным порядком редукции называется редукция самого левого $\beta$-редекса.
\end{definition}
"<Ленивые вычисления"> (ну, почти, в ленивых ещё есть меморизация)

\begin{definition}[Аппликативный порядок редукции]
    Самый левый из самых вложенных.
\end{definition}
"<Энергичные вычисления">

\begin{statement}
    Если нормальная форма существует, она может быть достигнута нормальным порядком редукции.
\end{statement}

\subsection{\texorpdfstring{Парадокс Карри}{Curry's paradox}}

\epigraph{Если это утверждение верно, то русалки существуют.}

Попробуем построить логику на основе $\lambda$-исчисления.
Введём комбнатор-импликацию, обозначим $(\supset)$. Введём M.P. и правила:
\begin{enumerate}
    \item $A \supset A$
    \item $(A \supset (A \supset B)) \supset (A \supset B)$
    \item $A =_{\beta} B$, тогда $A \supset B$
\end{enumerate}

Покажем, как в полученной логике можно доказать любое утверждение.
Введём обозначение: $Y_{\supset a} \equiv Y (\lambda t . t \supset a) =_{\beta} Y (\lambda t . t \supset a) \supset a$.

\begin{tabular}{lll}
    1) & $Y_{\supset a} \supset Y_{\supset a}$ & (схема аксиом) \\
    2) & $Y_{\supset a} \supset (Y_{\supset a} \supset a)$ & (можно доказать) \\
    3) & $(Y_{\supset a} \supset Y_{\supset a} \supset a) \supset (Y_{\supset a} \supset a)$ & (схема аксиом) \\
    4) & $Y_{\supset a} \supset a$ & (M.P.) \\
    5) & $(Y_{\supset a} \supset a) \supset Y_{\supset a}$ & (третье правило) \\
    6) & $Y_{\supset a}$ & (M.P.) \\
    7) & $a$ & (M.P.)
\end{tabular}

Получается, что данная логика противоречива.

\subsection{\texorpdfstring{Импликационный фрагмент ИИВ}{Implication fragment of intuitionistic logic}}

\begin{definition}[импликационный фрагмент ИИВ]
    Рассмотрим интуиционистское исчисление высказываний.
    \begin{enumerate}
        \item Введём схему аксиом:
        \[
            \infer{\Gamma, \varphi \vdash \varphi}{}
        \]
        \item Правило введения импликации:
        \[
            \infer{\Gamma \vdash \varphi \rightarrow \psi}{\Gamma, \varphi \vdash \psi}
        \]
        \item И правило удаления импликации:
        \[
            \infer{\Gamma \vdash \psi}{\Gamma \vdash \varphi \rightarrow \psi & \Gamma \vdash \varphi}
        \]
    \end{enumerate}

    Мы построили импликационный фрагмент ИИВ (и.ф.и.и.в).
\end{definition}

\begin{example} Докажем $\varphi \rightarrow \psi \rightarrow \varphi$:
\[
    \infer[(2)]
        { \vdash \varphi \rightarrow (\psi \rightarrow \varphi) }
        { \infer[(2)]
            { \varphi \vdash \psi \rightarrow \varphi }
            { \infer[(1)]
                { \varphi, \psi \vdash \varphi}
                {}
            }
        }
\]
\end{example}

\begin{theorem}
    И.ф.и.и.в полон в моделях Крипке, то есть $\Gamma \vdash \varphi$ т.и.т.т.,
    когда для любой модели крипке $C$ из $\Vdash_C \Gamma$ следует $\Vdash_C \varphi$.
\end{theorem}

\begin{proof}
    Рассмотрим модель Крипке вида $W = \left\{\Delta \mid \Gamma \subseteq \Delta, \Delta\text{ замкнуто относительно }\vdash\right\}$,
    $\Gamma \leq \Delta$ если $\Gamma \subseteq \Delta$.
    Индукцией по структуре $\varphi$ покажем, что $\Delta \Vdash \varphi$ т.и.т.т., когда $\Delta \vdash \varphi$:
    \begin{enumerate}
        \item Пусть $\varphi \equiv x$ "--- переменная. Тогда $\Gamma \vdash \varphi$ эквивалентно $x \in \Gamma$, что эквивалентно $\Vdash x$ (по определению).
        \item Пусть $\varphi \equiv \alpha \rightarrow \beta$.
        \begin{enumerate}[label=(\asbuk*)]
            \item Пусть $\Delta \vdash \varphi$.
            Рассмотрим такое $\Delta'$, что $\Delta \leq \Delta'$ и $\Delta' \Vdash \alpha$.
        \end{enumerate}
        ой всё \todo
    \end{enumerate}
\end{proof}

\begin{corollary}
    И.ф.и.и.в замкнут относительно выводимости.
\end{corollary}
Если некоторое утверждение выводится в ИИВ ($\vdash_{и} \varphi$) и содержит только импликации,
то оно выводится и в и.ф.и.и.в. ($\vdash_{и \rightarrow} \varphi$).

\section{\texorpdfstring{Просто типизированное $\lambda$-исчисление}{Simply typed lambda calculus}}

\begin{definition}[Тип]
    $T = \{\alpha, \beta, \gamma, \ldots\}$ "--- множество типов.
    $\sigma$, $\tau$ "--- метапеременные для типов.
    Если $\tau$, $\sigma$ "--- типы, то $\sigma \rightarrow \tau$ "--- тип.
    \begin{bnf}
    \[
        \Pi ::= T | \Pi \rightarrow \Pi | (\Pi)
    \]
    \end{bnf}
    $\left(\rightarrow\right)$ правоассоциативна.
\end{definition}

\begin{definition}[Контекст] Контекст "--- $\Gamma$.
\begin{gather*}
    \Gamma = \{ \Lambda_{1} : \sigma_{1};\ \Lambda_{2} : \sigma_{2}\ \ldots\ \Lambda_{n} : \sigma_{n} \} \\
    \abs{\Gamma} = \{ \sigma_{1},\ \sigma_{2}\ \ldots\ \sigma_{n} \} \\
    \mathrm{dom}\ \Gamma = \{ \Lambda_{1},\ \Lambda_{2}\ \ldots\ \Lambda_{n} \}
\end{gather*}
\end{definition}

\subsection{\texorpdfstring{Исчисление по Карри}{Curry-style}}

\begin{definition}[Типизируемость по Карри]
    Рассмотрим исчисление со следующими правилами:
    \begin{enumerate}
        \item $\infer[(x \notin \mathrm{dom}(\Gamma))]
            {\Gamma, x:\sigma \vdash x:\sigma}
            {}$
        \item $\infer[]
            {\Gamma \vdash M N : \tau}
            {\Gamma \vdash M:\sigma \rightarrow \tau & \Gamma \vdash N:\sigma}$
        \item $\infer[(x \notin \mathrm{dom}(\Gamma))]
            {\Gamma \vdash \lambda x . M : \sigma \rightarrow \tau}
            {\Gamma, x : \sigma \vdash M : \tau}$
    \end{enumerate}
    %\begin{gather*}
    %\infer[(x\ \cancel{\in}\ \mathrm{dom}(\Gamma))]
    %    {\Gamma, x:\sigma \vdash x:\sigma}
    %    {} \\
    %\infer[]
    %    {\Gamma \vdash M N : \tau}
    %    {\Gamma, M:\sigma \rightarrow \tau & \Gamma \vdash N:\sigma} \qquad
    %\infer[(x\ \cancel{\in}\ \mathrm{dom}(\Gamma))]
    %    {\Gamma \vdash \lambda x . M : \sigma \rightarrow \tau}
    %    {\Gamma, x : \sigma \vdash M : \tau}
    %\end{gather*}
    Если $\lambda$-выражение типизируется этими трёмя правилами, то говорят, что оно типизируется по Карри.
\end{definition}

\begin{lemma}[subject reduction]
    Если $\Gamma \vdash M : \sigma$ и $M \rightarrow_{\beta}N$, то $\Gamma \vdash N : \sigma$.
\end{lemma}

\begin{corollary}
    Если $\Gamma \vdash M : \sigma$ и $M \twoheadrightarrow_{\beta}N$, то $\Gamma \vdash N : \sigma$.
\end{corollary}

\begin{theorem}[Чёрча-Россера]
    Если $\Gamma \vdash M : \sigma$, $M \twoheadrightarrow_{\beta} N$ и $M \twoheadrightarrow_{\beta} P$, тогда найдётся $Q$, что
    $N \twoheadrightarrow_{\beta} Q$, $P \twoheadrightarrow_{\beta} Q$ и $\Gamma \vdash Q : \sigma$.
\end{theorem}

\begin{example} Несколько доказательств:
    \begin{enumerate}
        \item Докажем $\lambda x . x : \alpha \rightarrow \alpha$:
        \[
            \infer[(3)]
                {\vdash \lambda x . x : \alpha \rightarrow \alpha}
                { \infer[(1)]
                    {x : \alpha \vdash x : \alpha}
                    {}
                }
        \]

        \item Докажем $\lambda f . \lambda x . f x : (\alpha \rightarrow \beta) \rightarrow \alpha \rightarrow \beta$:
        \[
            \infer[(3)]
                { \vdash \lambda f . \lambda x . f x : (\sigma \rightarrow \tau) \rightarrow (\sigma \rightarrow \tau) }
                { \infer[(3)]
                    { f : \sigma \rightarrow \tau \vdash \lambda x . f x : \sigma \rightarrow \tau }
                    { \infer[(2)]
                        {f : \sigma \rightarrow \tau; x : \sigma \vdash f x : \tau}
                        {
                            \infer[(1)]{ \Gamma \vdash f : \sigma \rightarrow \tau }{} &
                            \infer[(1)]{ \Gamma \vdash x : \sigma }{}
                        }
                    }
                }
        \]

        \item $\Omega = (\lambda x . x x) (\lambda x . x x)$ не типизируемо:
            \todo % TODO
    \end{enumerate}
\end{example}

\begin{lemma}[Свойство subject expansion]
    Неверно, что если $M \rightarrow_{\beta} N$, $\Gamma \vdash N : \sigma$, то $\Gamma \vdash M : \sigma$.
\end{lemma}
Например, для $Ka\Omega$.

В общем случае тип не уникален, бывает, что одновременно $\vdash \lambda x . x : \alpha \rightarrow \alpha$ и $\vdash \lambda x . x : (\beta \rightarrow \beta) \rightarrow (\beta \rightarrow \beta)$.

\begin{definition}[Сильная нормализация] \label{strong-normalization}
    Назовём исчисление сильно-нормализуемым, если любая последовательность редукций неизбежно приводит к нормальной форме (не существует бесконечной последовательности $\beta$-редукций) .
\end{definition}

\begin{definition}[Слабая нормализация]
    Назовём исчисление слабо-нормализуемым, если для любого терма существует последовательность $\beta$-редукций, приводящая его к нормальной форме.
\end{definition}

\begin{theorem}[о сильной нормализации]
    Просто типизируемое $\lambda$-исчисление сильно нормализуемо.
    Любое просто типизируемое $\lambda$-выражение сильно нормализуемо.
\end{theorem}

%\todo\{ это к чему и о чём вообще?
%\begin{theorem}
%    Если $\nu = (\alpha \rightarrow \alpha) \rightarrow (\alpha \rightarrow \alpha)$ и $F : \nu \rightarrow \nu \rightarrow \nu$, то $F$ "--- расширенный полином.
%\end{theorem}
%
%\begin{theorem}
%    Рассмотрим полином $E(m,n) =
%    \begin{cases}
%        \text{Полином}(m,n) & m > 0, n > 0 \\
%        \text{Полином}(m)   & m > 0, n = 0 \\
%        \text{Полином}(n)   & m = 0, n > 0 \\
%        \text{Константа}    & m = 0, n = 0
%    \end{cases}$. \\
%    Если $\nu = (\alpha \rightarrow \alpha) \rightarrow (\alpha \rightarrow \alpha)$ и $F : \nu \rightarrow \nu \rightarrow \nu$, то $F$ "--- рассматриваемый полином.
%\end{theorem}
%\} // \todo

\subsection{\texorpdfstring{Исчисление по Чёрчу}{Church-style}}

\begin{definition}[Типизация по Чёрчу]
    \begin{bnf}
    \[
        \Lambda_{\xx} ::= x | \lambda x^{\sigma}.\Lambda_{\xx} | (\Lambda_{\xx}) | \Lambda_{\xx} \Lambda_{\xx}
    \]
    \end{bnf}
    Правила:
    \begin{enumerate}
        \item $\infer[(x \notin \mathrm{dom}(\Gamma))]
            {\Gamma, x:\sigma \vdash_{\xx} x:\sigma}
            {}$
        \item $\infer[]
            {\Gamma \vdash_{\xx} M N : \tau}
            {\Gamma \vdash_{\xx} M:\sigma \rightarrow \tau & \Gamma \vdash_{\xx} N:\sigma}$
        \item $\infer[(x \notin \mathrm{dom}(\Gamma))]
            {\Gamma \vdash_{\xx} \lambda x^{\sigma} . M : \sigma \rightarrow \tau}
            {\Gamma, x : \sigma \vdash_{\xx} M : \tau}$
    \end{enumerate}

\end{definition}

\begin{definition}
\[
    \abs{\Lambda_{\xx}} =
    \begin{cases}
        x                                   & \Lambda_{\xx} \equiv x \\
        \abs{\Lambda_{1}} \abs{\Lambda_{2}} & \Lambda_{\xx} \equiv \Lambda_{1} \Lambda_{2} \\
        \lambda x . \abs{\Lambda}           & \Lambda_{\xx} \equiv \lambda x^{\sigma} . \Lambda
    \end{cases}
\]
\end{definition}

\begin{lemma}[Subject reduction по Чёрчу]
    Пусть $\Gamma \vdash_{\xx} M : \sigma$ и $\abs{M} \rightarrow_{\beta} N$. \\
    Тогда найдётся такое $H$, что $\abs{H} = N$, $\Gamma \vdash_{\xx} H:\sigma$.
\end{lemma}

\begin{theorem}[Чёрча-Россера]
    Пусть $\Gamma \vdash_{\xx} M : \sigma$, $\abs{M} \twoheadrightarrow_{\beta} N$, $\abs{M} \twoheadrightarrow_{\beta} T$. \\
    Тогда найдётся такое $P$, что $\Gamma \vdash_{\xx} P : \sigma$,
            $N \twoheadrightarrow_{\beta} \abs{P}$ и $T \twoheadrightarrow_{\beta} \abs{P}$.
\end{theorem}

\begin{lemma}[Уникальность типов] \label{uniqueness}
    Если $\Gamma \vdash_\xx M : \gamma$ и $\Gamma \vdash_\xx M : \tau$, то $\sigma = \tau$.
\end{lemma}

Лемма \ref{uniqueness} показывает, чем исчисление по Чёрчу отличается исчислением по Карри.

\begin{theorem}[о стирании] \ 
    \begin{enumerate}
        \item Если $M \rightarrow_{\beta} N$ и $\Gamma \vdash_{\xx} M : \sigma$, то $\abs{M} \rightarrow_{\beta} \abs{N}$.
        \item Если $\Gamma \vdash_{\xx} M : \sigma$, то $\Gamma \vdash_{к} \abs{M} : \sigma$.
    \end{enumerate}
\end{theorem}

\begin{theorem}[о поднятии]
    Пусть $P \in \Lambda_{\xx}$, $M, N \in \Lambda_{\rr}$.
    \begin{enumerate}
        \item Если $M \rightarrow_{\beta} N$, $\abs{P} = M$, то найдётся такое $Q$, что $\abs{Q} = N$, $P \rightarrow_{\beta} Q$.
        \item Если $\Gamma \vdash_{\rr} M : \sigma$, то найдётся такое $P \in \Lambda_{\xx}$, что
                $\Gamma \vdash_{\xx} P : \sigma$, $\abs{P} = M$.
    \end{enumerate}
\end{theorem}

\subsection{\texorpdfstring{Изоморфизм Карри-Ховарда}{Curry-style}}

\todo Многими логиками замечалась связь между выражениями типизированного $\lambda$-исчисления и доказательствами выражений ИИВ.

\begin{center}
\begin{tabular}{|Sc|Sc|} \hline
    Правило вывода в ИИВ & Правило вывода в типизированном $\lambda$-исчислении \\ \hline

    $\infer{\Gamma \vdash \tau}{\Gamma \vdash \sigma \rightarrow \tau & \Gamma \vdash \sigma}$ &
    $\infer{\Gamma \vdash AB : \tau}{\Gamma \vdash A : \sigma \rightarrow \tau & \Gamma \vdash B : \sigma}$ \\

    $\infer{\Gamma \vdash \sigma \rightarrow \tau}{\Gamma, \sigma \vdash \tau}$ &
    $\infer{\Gamma \vdash \lambda x^\sigma . A : \sigma \rightarrow \tau}{\Gamma, x^\sigma : \sigma \vdash A : \tau}$ \\ \hline

    $\infer{\Gamma \vdash \varphi \with \psi}{\Gamma \vdash \varphi & \Gamma \vdash \psi}$ &
    $\infer{\Gamma \vdash \left<A,B\right> : \varphi \with \psi}{\Gamma \vdash A : \varphi & \Gamma \vdash B : \psi}$ \\

    $\infer{\Gamma \vdash \varphi}{\Gamma \vdash \varphi \with \psi}$ &
    $\infer{\Gamma \vdash \pi_1 R : \varphi}{\Gamma \vdash R : \varphi \with \psi}$ \\

    $\infer{\Gamma \vdash \psi}{\Gamma \vdash \varphi \with \psi}$ &
    $\infer{\Gamma \vdash \pi_2 R : \psi}{\Gamma \vdash R : \varphi \with \psi}$ \\ \hline

    $\infer{\Gamma \vdash \varphi \vee \psi}{\Gamma \vdash \varphi}$ &
    $\infer{\Gamma \vdash \inj_1 A : \varphi \vee \psi}{\Gamma \vdash A : \varphi}$ \\

    $\infer{\Gamma \vdash \varphi \vee \psi}{\Gamma \vdash \psi}$ &
    $\infer{\Gamma \vdash \inj_2 A : \varphi \vee \psi}{\Gamma \vdash A : \psi}$ \\

    $\infer{\Gamma \vdash \varphi \vee \psi \rightarrow \pi}
        {\Gamma \vdash \varphi \rightarrow \pi & \Gamma \vdash \psi \rightarrow \pi}$ &
    $\infer{\Gamma \vdash \case{T}{A}{B} : \pi}{\Gamma \vdash T : \varphi \vee \psi &
        \Gamma \vdash A : \varphi \rightarrow \pi & \Gamma \vdash B : \psi \rightarrow \pi}$ \\ \hline
\end{tabular}
\end{center}

\todo\ Пояснения значений инъекций, проекций и case; примеры.

\begin{center}
\begin{tabular}{|Sc|Sc|} \hline
    Интуиционистская логика & $\lambda$-исчисление \\ \hline
    выражение & тип \\
    доказательство & терм (программа) \\
    предположение & свободная переменная \\
    импликация & абстракция (функция) \\ \hline
\end{tabular}
\end{center}

\begin{theorem}[об изоморфизме Карри-Ховарда] \ 
    \begin{enumerate}
        \item Пусть $\Gamma \vdash \sigma$ "--- и.ф.и.и.в., тогда найдётся такой терм M,
            что $\Delta \vdash_{\xx} M : \sigma$, где $\Delta=\left\{ \left(x^\varphi : \varphi \right) \mid \varphi \in \Gamma \right\}$.
        \item Пусть $\Delta \vdash_{\xx} M : \sigma$, тогда $\abs{\Delta} \vdash \sigma$.
    \end{enumerate}
\end{theorem}

%% долг с предыдущей лекции
\begin{proof} \todo
    %есть в Curry-Howard Isomorphism, стр 75

    %Рассмотрим $x_\varphi$, $\varphi \in \Gamma$.
    %$\Gamma = \{\sigma_{1}, \sigma_{2} \ldots\}$,
    %$\uparrow \!\! \Gamma = \Delta = \{x_{1}:\sigma_{1}, x_{2}:\sigma_{2}, \ldots \}$.
    %Индукция по сложности доказательства $\sigma$: \\
    %База: \infer{\Gamma, \varphi \vdash \varphi}{}
    %\begin{enumerate}
    %    \item $\varphi \in \Gamma$. Тогда \infer{\Delta \vdash x_\varphi : \varphi}{}
    %    \item $\varphi \notin \Gamma$. Тогда \infer{\Delta, x_\varphi : \varphi \vdash x_\varphi : \varphi}{}, $x_\varphi$ "--- новая переменная.
    %\end{enumerate}
    %Переход: \infer{\Gamma \vdash \varphi \rightarrow \psi}{\Gamma, \varphi \vdash \psi}.
    %По индукционному предположению $\Delta, y : \varphi \vdash M : \psi$.
    %\begin{enumerate}
    %    \item $\varphi \in \Gamma$. Тогда
    %    \infer{\Delta \setminus \left\{x_\varphi\right\} \vdash \lambda x_\varphi . M : \varphi \rightarrow \psi}{\Delta \vdash M : \psi}.
    %    \item $\varphi \notin \Gamma$. Тогда
    %    \infer{\Delta \vdash \lambda x_\varphi . M : \varphi \rightarrow \psi}{\Delta, x_\varphi : \varphi \vdash M : \psi}.
    %\end{enumerate}
\end{proof}

\section{\texorpdfstring{Задачи в $\lambda$-исчислении}{Link to programming}}
\epigraph{Помните, что в $\lambda$-исчислении нет смысла? Здесь смысл отрицательный, скорее.}{Д.Г.}

В $\lambda$-исчислении выделяют 3 задачи:
\begin{enumerate}[label=(\asbuk*)]
    \item Проверка типа: верно ли $\Gamma \vdash M : \sigma$?
    \item Вывод типа: $? \vdash M : ?$
    \item Обитаемость типа: $? \vdash ? : \sigma$
\end{enumerate}

В этом разделе будем рассматривать задачу вывода типа.

\subsection{\texorpdfstring{Вывод типа}{Type deduction}}

\begin{definition}[Алгебраический терм]
    \begin{gather*}
        \begin{bnf}
            A ::= x | f\left(A, \ldots, A\right)
        \end{bnf} \\
        (x \in X)
    \end{gather*}
\end{definition}

Уравнение в алгебраических термах: $A = A$.

\begin{definition}[$S$-подстановка]
    \[
        S : X \rightarrow A
    \]
    Причём $S$ "--- id почти везде. (везде кроме конечного количества)
\end{definition}

\begin{definition}[Естественное обобщение]
    Естественное обобщение "--- такая подстановка $S : A \rightarrow A$, что
    $S\left(f\left(A_1, \dots, A_n\right)\right) = f\left(S(f_1), \ldots, S(f_n)\right)$
\end{definition}

\begin{definition}[Унификатор]
    $S$ "--- унификатор (решение уравнения) $P=Q$, если $S(P)=S(Q)$.
\end{definition}
Задача решения уравнение в алгебраических термах "--- унификация.

\begin{definition}[Композиция]
    $(S \circ T)(A) = S(T(A))$
\end{definition}

\begin{definition}[Частный случай]
    $T$ "--- частный случай $U$, если существует такое $S$, что $T = S \circ U$.
\end{definition}

\begin{definition}[Наибольший общий унификатор]
    Наибольший общий унификатор $U$ для уравнения $A=B$ "--- такой унификатор, что:
    \begin{enumerate}
        \item $U(A)=U(B)$.
        \item Любой другой унификатор "--- частный случай $U$.
    \end{enumerate}
\end{definition}

\begin{definition}[Несовместная система]
    Назовём систему несовместной, если выполнено одно из условий:
    \begin{enumerate}
        \item в ней есть уравнение вида $f(\ldots)=g(\ldots)$
        \item в ней есть уравнение вида $x = \ldots x \ldots$
    \end{enumerate}
\end{definition}

\begin{definition}[Эквивалентные системы]
    Назовём две системы эквивалентными, если они имеют одинаковые решения.
\end{definition}

\begin{statement}
    Для любой системы
    \[
        \begin{cases} A_1 = B_1 \\ \vdots \\ A_n = B_n \end{cases}
    \]
    найдётся эквивалентная ей система из одного уравнения:
    \[
        f(A_1, \ldots, A_n) = f(B_1, \ldots, B_n)\text{,}
    \]
    где $f$ "--- новый символ.
\end{statement}

\begin{definition}[Разрешённая система]
    Назовём систему разрешённой, если:
    \begin{enumerate}
        \item все уравнения имеют вид $x = A$;
        \item все переменные в левой части встречаются однократно.
    \end{enumerate}
\end{definition}

Решение по системе в разрешённой форме строится так:
По системе в разрешённой форме мы можем построить решение $S$, определив $S(x_i) = A_i$ для каждого $i$.
\todo

\begin{statement}
    Построенный по системе в разрешённой форме унификатор $S$ "--- наибольший общий унификатор.
\end{statement}

\begin{statement}
    Несовместная система не имеет решений.
\end{statement}

Рассотрим следующие преобразования, которые не меняют свойства системы:
\begin{center}
\begin{tabular}{Sl Sl Sl}
    Выражения                         & Условия             & Новые выражения \\ \hline
    $T=x$, $T$ не переменная          &                     & $x=T$ \\ \hline
    $T=T$                             &                     & убрать это уравнение \\ \hline
    $f(A_1, \ldots A_n) = g(B_1, \ldots B_n)$
                                      & $f=g$               & $ A_1 = B_1 \ldots A_n = B_n$ \\ \cline{2-3}
                                      & $f \neq g$          & система несовместна \\ \hline
    $x=T$, $R=S$, $x$ входит в $S$ или $T$
                                      & $T$ не содержит $x$ & $x=T$, $R\left[x\coloneqq T\right]=S\left[x\coloneqq T\right]$ \\ \cline{2-3}
                                      & $T$ содержит $x$    & система несовместна \\ \hline
\end{tabular}
\end{center}

\begin{statement}
    Последовательное применение правил либо за конечное число шагов приведёт систему в разрешённый вид, либо сделает её несовместной.
\end{statement}

\begin{proof} \todo
    Пусть $(n_1, n_2, n_3)$ "--- характеристика системы, где
    $n_1$ "--- количество переменных не входящих слева в систему слева от знака равенства только один раз,
    $n_2$ "--- общее количество вхождений функциональных символов в $S$,
    $n_3$ "--- количество выражений вида $x=x$ или $T=x$.
    Каждое преобразование уменьшает эту тройку (если сравнивать лексикографически).
\end{proof}

\begin{theorem}
    Задача вывода типа в $\lambda$-исчислении разрешима.
\end{theorem}

\begin{proof}
    Опишем алгоритм. \\
    Пусть нам дан $\lambda$-терм $M$. Рекурсивно построим по нему систему уравнений $E_m$:
    \begin{center}
    \begin{tabular}{Sl Sl Sl} \hline
        $M \equiv x$ & $E_m=\set{}$ & $\tau_m=\alpha$ "--- новая переменная. \\ \hline
        $M \equiv PR$ & $E_m=E_p \cup E_r \cup \set{\tau_p=\tau_r\rightarrow\pi}$ & $\tau_m=\pi$ \\ \hline
        $M \equiv \lambda x . P$ & $E_m=E_p$ & $\tau_m=\tau_x\rightarrow\tau_p$ \\ \hline
    \end{tabular}
    \end{center}
    Решим построенную систему уравнений.\\
    Можно показать, что алгоритм корректный.
\end{proof}

\begin{example}
    \todo %todo
\end{example}

\subsection{\texorpdfstring{Про ложь}{About false}}

\todo\ послушать запись. %todo

\section{\texorpdfstring{Система $F$}{System F}}

Обычное $\lambda$-исчисление позволяет слишком много, просто-типизированное "--- слишком мало ($(-1)$ не выразим). Хотелось бы золотую середину.

\subsection{\texorpdfstring{Интуиционистское исчисление предикатов второго порядка}{Second order intuitionistic logic}}

\begin{definition}
    \begin{bnf}
    \[
        \Phi ::= (\Phi) | p | \Phi \rightarrow \Phi | \forall p . \Phi \color{gray}
            \underbrace{| \exists p . \Phi | \bot | \Phi \with \Phi | \Phi \vee \Phi}_{\text{не существенные}}
    \]
    \end{bnf}
    Введение кванторов:
    \begin{gather*}
        \infer[p \notin \FV(\Gamma)]{\Gamma \vdash \forall p . \varphi}{\Gamma \vdash \varphi} \qquad
        \infer{\Gamma \vdash \exists p . \varphi}{\Gamma \vdash \varphi \left[p \coloneqq \psi\right]}
    \end{gather*}
    Удаление кванторов:
    \begin{gather*}
        \infer{\Gamma \vdash \varphi \left[p \coloneqq \sigma\right]}{\Gamma \vdash \forall p . \varphi} \qquad
        \infer[p \notin \FV(\Gamma, \psi)]{\Gamma \vdash \psi}{\Gamma \vdash \exists p . \varphi && \Gamma, \varphi \vdash \psi}
    \end{gather*}
    Последние четыре связки можно выразить через первые:
    \begin{align*}
        \bot & \equiv \forall p . p \\
        \varphi \with \psi & \equiv \forall a . ((\varphi \rightarrow \psi \rightarrow a) \rightarrow a) \\
        \varphi \vee \psi & \equiv \forall a . (\varphi \rightarrow a) \rightarrow (\psi \rightarrow a) \rightarrow a \\
        \exists x . \tau & \equiv \forall a . (\forall x . \tau \rightarrow a) \rightarrow a
    \end{align*}
\end{definition}

\subsection{\texorpdfstring{Система $F$}{System F}}
\begin{definition}[Тип в системе $F$]
\[
    \tau =
    \begin{cases}
        \alpha, \beta, \gamma, \ldots & \text{(атомарный тип)} \\
        \tau \rightarrow \sigma \\
        \forall \alpha . \tau & \text{($\alpha$ "--- переменнная)}
    \end{cases}
\]
\end{definition}

\begin{definition}[Исчисление по Чёрчу в системе $F$]
    \begin{bnf}
        \begin{gather*}
            \mathbf\Lambda ::= x | \lambda p^\alpha . \mathbf\Lambda | \mathbf\Lambda \mathbf\Lambda | (\mathbf\Lambda)
            | \Lambda \alpha . \mathbf\Lambda | \mathbf\Lambda \tau
        \end{gather*}
    \end{bnf}
    $\Lambda \alpha . \mathbf\Lambda$ "--- типовая (полиморфная абстракция), $\mathbf\Lambda \tau$ "--- применение типа.

    Правила вывода:
    \begin{gather*}
        \infer[x \notin \mathrm{dom}(\Gamma)]{\Gamma, x : \sigma \vdash x : \sigma}{} \\
        \infer{\Gamma \vdash MN : \sigma}{\Gamma \vdash M : \tau \rightarrow \sigma & \Gamma \vdash N : \tau} \qquad
        \infer[(x \notin \mathrm{dom}(\Gamma))]{\Gamma \vdash \lambda x^\tau . M : \tau \rightarrow \sigma}{\Gamma, x : \tau \vdash M : \sigma} \\
        \infer[x \in \FV(\Gamma)]{\Gamma \vdash \Lambda \alpha . M : \forall \alpha : \sigma}{\Gamma \vdash M : \sigma} \qquad
        \infer[\text{(подстановка типа)}]{\Gamma \vdash M \tau : \sigma [\alpha := \tau]}{\Gamma \vdash M : \forall \alpha . \sigma}
    \end{gather*}
\end{definition}

\begin{example} Левая проекция: \\
    \begin{tabular}{l l l}
        & Просто типизированное $\lambda$-исчисление & Система $F$ \\
        Тип & $\pi_1:\alpha\with\beta\rightarrow\alpha$ & $\pi_1:\forall \alpha . \forall \beta . \alpha \with \beta \rightarrow \alpha$ \\
        Выражение & $\pi_1 = \lambda p . p T$ & $\pi_1 = \Lambda \varphi . \Lambda \psi . \lambda p^{\varphi\with\psi} . p T$
    \end{tabular}
\end{example}

\begin{definition}[$\beta$-редукция в $F$] \ 
    \begin{enumerate}[label=(\asbuk*)]
        \item Типовая редукция: $\left(\Lambda \alpha . M^\sigma\right) \tau \rightarrow_\beta M[\alpha:=\tau] : \sigma[\alpha := \tau]$
        \item Классическая $\beta$-редукция: $\left(\lambda x^\sigma . M\right)^{\sigma \rightarrow \tau} X \rightarrow_\beta M [x:=X] : \tau$
    \end{enumerate}
\end{definition}

\begin{theorem}[Изоморфизм Карри-Ховарда]
    $\Gamma \vdash_F M :\tau$ т.и.т.т., когда $\abs{\Gamma} \vdash \tau$ в интуиционистском исчислении предикатов второго порядка.
\end{theorem}

\begin{theorem}
    $F$ \hyperref[strong-normalization]{сильно нормализуемо}.
\end{theorem}

\todo 

\subsection{\texorpdfstring{Экзистенциальные тип}{Existential types}}

Допустим, у нас есть абстрактный тип данных "<Стек">: \\
\begin{tabular}{l l}
    empty & $: \alpha$ \\
    push  & $: \alpha\with\nu\rightarrow\alpha$ \\
    pop   & $: \alpha\rightarrow\alpha\with\nu$ \\
\end{tabular} \\
Можно попробовать сказать это так: "<$\mathrm{stack} :
    \alpha \with (\alpha\with\nu\rightarrow\alpha) \with (\alpha\rightarrow\alpha\with\nu)$">.
Но проблема в том, что у нас есть только интерфейс стека, а не его реализация. Поэтому лучше будет сказать так:
    $\exists \alpha . \alpha \with (\alpha\with\nu\rightarrow\alpha) \with (\alpha\rightarrow\alpha\with\nu)$.
То есть существует какое-то $\alpha$, реазизовывающее требуемый интерфейс.

По аналогии с правилом удаления квантора существования, можно определить правила вывода для выражений экзистенциальных типов:
\begin{gather*}
    \infer{\Gamma\vdash (\pack{M, \theta}{\exists \alpha . \varphi}) : \exists \alpha.\varphi}
        {\Gamma \vdash M : \varphi[\alpha := \theta]} \qquad
    \infer[\alpha \notin \FV(\Gamma, \psi)]
        {\Gamma \vdash~\abstype{\alpha}{x:\varphi}{M}{N:\psi}}
        {\Gamma \vdash M : \exists \alpha . \varphi && \Gamma, x : \varphi \vdash N : \psi}
\end{gather*}
\begin{example} \todo нормально написать пример со стеком.
\end{example}
%\begin{lstlisting}[language=ML]
%type 'a stack
%val empty : 'a stack
%val push  : 'a * 'a stack -> 'a stack
%val pop   : 'a stack -> 'a * 'a stack
%\end{lstlisting}
%-----------------------------------------
%\begin{lstlisting}[language=ML]
%type stack
%val empty : stack
%val push : int * stack -> stack
%val pop : stack -> int * stack
%\end{lstlisting}
Если вспомнить, что квантор существования выразим через квантор всеобщности
($\exists \alpha . x \equiv \forall \beta . (\forall \alpha . x \rightarrow \beta) \rightarrow \beta$),
то можно попытаться записать типы выражений \textbf{pack} и \textbf{abstype} через квантор существования и выразить их без расширения языка.
\begin{gather*}
    \pack{M, \theta}{\exists \alpha . \varphi} = 
        \Lambda \beta . \lambda x^{\forall \alpha . \varphi \rightarrow \beta} . (x \theta M) \\
    \abstype{\alpha}{x : \varphi}{M}{N^\psi} =
        M \psi (\Lambda \alpha . \lambda x ^ \varphi . N)
\end{gather*}
Можно показать, что $\abstype{\alpha}{x:\sigma}{(\pack{M,\tau}{\exists\alpha .\sigma})}{N}\twoheadrightarrow_\beta N[\alpha\coloneqq\tau][x\coloneqq M]$.
\todo
\begin{example} \todo
\end{example}

\begin{statement}
    $F$ сильно нормализуемо.
\end{statement}

\begin{statement}
    $F$ неразрешима.
\end{statement}
Ни одна из задач $\lambda$-исчисления в системе $F$ не разрешима, даже задача проверки типизации.
Доказать это можно через сведение к проблеме останова.

Итак, мы попытались добавить к типизированному $\lambda$-исчислению абстрактные типы данных, и получили слишком сложный язык.
Давайте попробуем немного его упростить, чтобы с ним можно было работать.

\subsection{\texorpdfstring{Типовая система Хиндли-Милнера}{\todo}}

\begin{definition}[Ранг типа]
\[
    \rank(\tau) =
    \begin{cases}
        \max(\rank(\sigma)+1, \rank(\rho)) & \tau \equiv \sigma \rightarrow \rho\text{, если }\sigma\text{ содержит }\forall \\
        \rank(\rho) & \tau \equiv \sigma \rightarrow \rho\text{, если }\sigma \text{ не содержит } \forall \\
        0 & \tau \equiv \alpha \\
        \max(\rank(\rho), 1) & \tau \equiv \forall \alpha . \rho
    \end{cases}
\]
\end{definition}

\begin{definition} \ \\
    Тип (монотип) "--- выражение в грамматике $ \begin{bnf} \tau ::= \alpha | \tau \rightarrow \tau | (\tau) \end{bnf} $. \\
    Типовая схема (политип) "--- выражение в грамматике $ \begin{bnf} \sigma ::= \tau | \forall \alpha . \sigma \end{bnf} $.
\end{definition}

\begin{statement}
    $\rank(\tau) = 0$, $\rank(\sigma) = 1$.
\end{statement}

\begin{example} Ранг экзистенциального типа
\begin{align*}
    \rank(\exists \alpha . \beta) &= \rank(\forall \gamma . (\forall \alpha . \beta \rightarrow \gamma) \rightarrow \gamma) \\
    &= \max(\rank((\forall \alpha . \beta \rightarrow \gamma) \rightarrow \gamma), 1) \\
    &= \max(\max(\rank(\forall \alpha . \beta \rightarrow \gamma) + 1, \rank(\gamma)), 1) \\
    &= \max(\max(2, 0), 1) = 2
\end{align*}
\end{example}

\begin{definition}
    $\sigma_1$ "--- подтип $\sigma_2$, если существует подстановка
            $[\alpha_1 \coloneqq \theta_1, \alpha_2 \coloneqq \theta_2 \ldots \alpha_n \coloneqq \theta_n]$:
    \begin{enumerate}
        \item $\sigma_1 = \forall \beta_1 \ldots \forall \beta_k . \tau [\alpha_1 \coloneqq \theta_1 \ldots \alpha_n := \theta_n]$,
            $\alpha_i$ не входит свободно в $\theta_j$.
        \item $\sigma_2 = \forall \alpha_1 \ldots \forall \alpha_n \tau$
    \end{enumerate}
\end{definition}

\begin{definition}[Правила вывода в системе Хиндли-Милнера]
\begin{gather*}
    \infer[\text{(Аксиома)}]{\Gamma, x : \sigma \vdash x : \sigma}{} \qquad
    \infer[\text{(Уточнение), $\sigma_1$ "--- подтип $\sigma$}]{\Gamma \vdash e : \sigma'}{\Gamma \vdash e : \sigma} \\
    \infer[\text{(Обобщение), } \sigma\notin\FV(\Gamma)]{\Gamma \vdash e : \forall \alpha . \sigma}{\Gamma \vdash e : \sigma} \qquad
    \infer[\text{(Абстракция)}]{\Gamma \vdash \lambda x . e : \tau' \rightarrow \tau}{\Gamma, x : \tau' \vdash e : \tau} \\
    \infer[\text{(Подстановка, применение)}]
        {\Gamma \vdash e e' : \tau}{\Gamma \vdash e : \tau' \rightarrow \tau && \Gamma \vdash e' : \tau'} \qquad
    \infer[\text{(Let)}]
        {\Gamma \vdash \lett{x=e}{e'=\tau}}
        {\Gamma \vdash e : \sigma && \Gamma \setminus \set{x}, x : \sigma \vdash e' : \tau}
\end{gather*}
\end{definition}
\todo попросить у Д.Г. оригинальную статью.

Грамматика выражения выглядит следующим образом:
\[
\begin{bnf}
    \Lambda ::= x | \lambda x . \Lambda | \Lambda \Lambda | (\Lambda) | \lett{x=\Lambda}{\Lambda}
\end{bnf}
\]

\subsection{\texorpdfstring{Алгоритм W}{Algorithm W}}
\begin{statement}
    Задача вывода типа в Х-М разрешима.
\end{statement}
Обозначения: $\overline{\Gamma(\tau)} = \forall \alpha_1 \ldots \forall \alpha_n . \tau$, где $\alpha_1 \ldots \alpha_n \notin \FV(\Gamma)$
(замыкание всех несвязанных переменных); $\Gamma_x = \Gamma \setminus \set{x}$.

Определим $W(\Gamma, e) = (S, \tau)$, принимающую контекст и выражение и возвращающую такую подстановку и тип,
что $S(\Gamma) \vdash e : \tau$.
\begin{center}
\begin{tabular}{Sl Sl Sl} \hline
    $e \equiv x$
        & $x : \forall \alpha_1 \ldots \alpha_k . \tau' \in \Gamma$
        & $S'=\mathrm{Id}$ \\ \hline
    $e \equiv e_1 e_2$
        & \begin{tabular}[t]{@{}Sl@{}}
            $W(\Gamma, e_2) = (S_2, \tau_2)$ \\
            $W(S_2(\Gamma), e_1) = (S_1, \tau_1)$ \\
            $U(S_1(\tau_1), \tau_2 \rightarrow \beta) = V$, где $\beta$ "--- новый тип
        \end{tabular}
        & $(V(S_1 \circ S_2), V(\beta))$ \\ \hline
    $e\equiv\lambda x . e$
        & $W(\Gamma_x \cup \set{x : \beta}, e) = (S_1, \tau_1)$, $\beta$ "--- новый тип
        & $(S_1, S_1(\beta \rightarrow \tau_1))$. \\ \hline
    $e\equiv\lett{x=e_1}{e_2}$
        & \begin{tabular}[t]{@{}Sl@{}}
            $W(\Gamma, e_1) = (S_1, \tau_1)$ \\
            $W(S_1(\Gamma_x \cup \{x : S_1(\overline{\Gamma(\tau_1)})\}), e_2) = (S_2, \tau_2)$
        \end{tabular}
        & $(S_2 \circ S_1, \tau_2)$ \\ \hline
\end{tabular}
\end{center}
\todo примеры

\section{\texorpdfstring{Линейные и уникальные типы}{Linear and unique types}}

Пусть $A \rightarrow_\beta A'$.
С одной стороны, порядок редукции не важен, % TODO картинка
$(\lambda x . x x) A \rightarrow_\beta (\lambda x . x x) A' \rightarrow_\beta A' A'$
и $(\lambda x . x x) A \rightarrow_\beta A A \rightarrow_\beta A' A \rightarrow_\beta A' A'$.
По теореме \nameref{church-rosser} нормальная форма единственна, если существует.
С другой стороны, реальный мир на самом деле не таков, в нём есть побочные эффекты.
% A(b(), unique_ptr<C>(new C))

\subsection{\texorpdfstring{Комбинаторы}{Combinators}}

Рассмотрим комбинаторные логики Моисея Шейнфинкеля и Хаскелла Карри:

Моисей Шейнфинкель:
\begin{center}
\begin{tabular}{Sc Sl}
    $I$ & Identität \\
    $K$ & Konstanz \\
    $S$ & VerSchmelzung \\
    $T$ & VerTauschung
\end{tabular}
\end{center}

Хаскел Карри:
\begin{align*}
    B &= \lambda x y z . x (y z) \\
    C &= \lambda x y z . x z y \\
    K &= \lambda x y . x \\
    W &= \lambda x y . x y y
\end{align*}


\begin{proof}[Докажем \ref{SK-basis}]
    Определим $T : \Lambda \rightarrow \Lambda_{SK} \cup \left\{ \text{свободные переменные} \right\}$:
    \begin{align*}
        T[x] &= x \\
        T[A B] &= T[A]\ T[B]  \\
        T[\lambda x . P] &= K\ T[P] \text{, если $x \notin FV(P)$} \\
        T[\lambda x . x] &= I =_\beta SKK \\
        T[\lambda x . A B] &= S\ T[\lambda x . A]\ T[\lambda x . B] \\
        T[\lambda x . \lambda y . A] &= T[\lambda x . T[\lambda y . A]]
    \end{align*}

    $T[\lambda\text{-выражение}]$ завершается и не содержит абстракций.
    Можно показать, что $T[A] =_\beta A$.
\end{proof}

Можно доказать аналогичную теорему для комбинаторов $B$, $C$, $K$, $W$, выразив через них $S$ и $K$:
$K = K$, $S = B (BW) (BBC)$, $I = CKK$.

А теперь выведем типы у $S$ и $K$:
\begin{align*}
    S &= \lambda x y z . x z (y z) : (\alpha \rightarrow \beta \rightarrow \gamma) \rightarrow
        (\alpha \rightarrow \beta) \rightarrow (\alpha \rightarrow \gamma) \\
    K &= \lambda x y . x : \alpha \rightarrow \beta \rightarrow \alpha
\end{align*}
Это похоже на вторую и первую схемы аксиом в ИИВ.

Базис $BCKWI$ по изоморфизму Карри-Ховарда порождает интуиционистскую логику.
Можно рассматривать исчисления, прорджённые базисами $BCKI$ и $BCI$.
\begin{align*}
    I &: \alpha \rightarrow \alpha \\
    B &: (\alpha \rightarrow \beta) \rightarrow (\beta \rightarrow \gamma) \rightarrow (\alpha \rightarrow \gamma) \\
    C &: (\alpha \rightarrow \beta \rightarrow \gamma) \rightarrow (\beta \rightarrow \alpha \rightarrow \gamma) \\
    K &: \alpha \rightarrow \beta \rightarrow \alpha \\
    W &: (\alpha \rightarrow \alpha \rightarrow \beta) \rightarrow (\alpha \rightarrow \beta)
\end{align*}

Сейчас мы выпишем какое-то исчисление.
\[
    \infer{\Gamma \vdash A \times B}{\Gamma \vdash A && \Gamma \vdash B} \qquad
    \infer{\Gamma \vdash A}{\Gamma \vdash A \times B} \qquad
    \infer{\Gamma \vdash B}{\Gamma \vdash A \times B}
\]

\todo{}шенька

\subsection{\texorpdfstring{Линейные высказывания}{Linear statements}}

\begin{definition}
    \begin{bnf}
    \[
        T ::= x | T \multimap T | T \otimes T | T \with T | T \oplus T | \oc T
    \]
    \end{bnf}
\end{definition}

Контексты двух сортов. $\left<A\right>$ "--- линейный, $\left[ A \right]$ "--- интуиционистский.

\begin{gather*}
    \infer{\left<A\right> \vdash A}{} \qquad
    \infer{\left[A\right] \vdash A}{} \qquad
    \infer{\Delta, \Gamma \vdash A}{\Gamma, \Delta \vdash A} \qquad
    \infer{\Gamma, \left[A\right] \vdash B}{\Gamma, \left[A\right], \left[A\right] \vdash B} \qquad
    \infer{\Gamma, \left[A\right] \vdash B}{\Gamma \vdash B} \\
\end{gather*}

Докажем закон Де-Моргана:
\[
    \infer{\left<\oc (A \with B)\right> \vdash \oc A \otimes \oc B} {
        \infer{\left<\oc (A \with B)\right> \vdash \oc (A \with B)}{}
        &&
        \infer{\left[A \with B\right] \vdash \oc A \otimes \oc B} {
            \infer{\left[A \with B, A \with B\right] \vdash \oc A \otimes \oc B}{
                \infer{\left[A \with B\right] \vdash \oc A} {
                    \infer{\left[A \with B\right] \vdash A} {
                        \infer{\left[A \with B \right] \vdash A \with B}{}
                    }
                }
                &&
                \infer{\left[A \with B\right] \vdash \oc B} {
                    \infer{\left[A \with B\right] \vdash B} {
                        \infer{\left[A \with B \right] \vdash A \with B}{}
                    }
                }
            }
        }
    }
\]

Вложение интуиционистских связок в нашу логику:
\begin{align*}
    A \rightarrow B &= \oc A \multimap B \\
    A \times B      &= A \with B \\
    A + B           &= \oc A \oplus \oc B
\end{align*}

Также можно вкладывать $A \times B$ как $\oc A \otimes \oc B$.


\end{document}
