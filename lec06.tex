\subsection{\texorpdfstring{Экзистенциальные тип}{Existential types}}

Допустим, у нас есть абстрактный тип данных "<Стек">: \\
\begin{tabular}{l l}
    empty & $: \alpha$ \\
    push  & $: \alpha\with\nu\rightarrow\alpha$ \\
    pop   & $: \alpha\rightarrow\alpha\with\nu$ \\
\end{tabular} \\
Можно попробовать сказать это так: "<$\mathrm{stack} :
    \alpha \with (\alpha\with\nu\rightarrow\alpha) \with (\alpha\rightarrow\alpha\with\nu)$">.
Но проблема в том, что у нас есть только интерфейс стека, а не его реализация. Поэтому лучше будет сказать так:
    $\exists \alpha . \alpha \with (\alpha\with\nu\rightarrow\alpha) \with (\alpha\rightarrow\alpha\with\nu)$.
То есть существует какое-то $\alpha$, реазизовывающее требуемый интерфейс.

По аналогии с правилом удаления квантора существования, можно определить правила вывода для выражений экзистенциальных типов:
\begin{gather*}
    \infer{\Gamma\vdash (\pack{M, \theta}{\exists \alpha . \varphi}) : \exists \alpha.\varphi}
        {\Gamma \vdash M : \varphi[\alpha := \theta]} \qquad
    \infer[\alpha \notin \FV(\Gamma, \psi)]
        {\Gamma \vdash~\abstype{\alpha}{x:\varphi}{M}{N:\psi}}
        {\Gamma \vdash M : \exists \alpha . \varphi && \Gamma, x : \varphi \vdash N : \psi}
\end{gather*}
\begin{example} \todo нормально написать пример со стеком.
\end{example}
%\begin{lstlisting}[language=ML]
%type 'a stack
%val empty : 'a stack
%val push  : 'a * 'a stack -> 'a stack
%val pop   : 'a stack -> 'a * 'a stack
%\end{lstlisting}
%-----------------------------------------
%\begin{lstlisting}[language=ML]
%type stack
%val empty : stack
%val push : int * stack -> stack
%val pop : stack -> int * stack
%\end{lstlisting}
Если вспомнить, что квантор существования выразим через квантор всеобщности
($\exists \alpha . x \equiv \forall \beta . (\forall \alpha . x \rightarrow \beta) \rightarrow \beta$),
то можно попытаться записать типы выражений \textbf{pack} и \textbf{abstype} через квантор существования и выразить их без расширения языка.
\begin{gather*}
    \pack{M, \theta}{\exists \alpha . \varphi} = 
        \Lambda \beta . \lambda x^{\forall \alpha . \varphi \rightarrow \beta} . (x \theta M) \\
    \abstype{\alpha}{x : \varphi}{M}{N^\psi} =
        M \psi (\Lambda \alpha . \lambda x ^ \varphi . N)
\end{gather*}
Можно показать, что $\abstype{\alpha}{x:\sigma}{(\pack{M,\tau}{\exists\alpha .\sigma})}{N}\twoheadrightarrow_\beta N[\alpha\coloneqq\tau][x\coloneqq M]$.
\todo
\begin{example} \todo
\end{example}

\begin{statement}
    $F$ сильно нормализуемо.
\end{statement}

\begin{statement}
    $F$ неразрешима.
\end{statement}
Ни одна из задач $\lambda$-исчисления в системе $F$ не разрешима, даже задача проверки типизации.
Доказать это можно через сведение к проблеме останова.

Итак, мы попытались добавить к типизированному $\lambda$-исчислению абстрактные типы данных, и получили слишком сложный язык.
Давайте попробуем немного его упростить, чтобы с ним можно было работать.

\subsection{\texorpdfstring{Типовая система Хиндли-Милнера}{\todo}}

\begin{definition}[Ранг типа]
\[
    \rank(\tau) =
    \begin{cases}
        \max(\rank(\sigma)+1, \rank(\rho)) & \tau \equiv \sigma \rightarrow \rho\text{, если }\sigma\text{ содержит }\forall \\
        \rank(\rho) & \tau \equiv \sigma \rightarrow \rho\text{, если }\sigma \text{ не содержит } \forall \\
        0 & \tau \equiv \alpha \\
        \max(\rank(\rho), 1) & \tau \equiv \forall \alpha . \rho
    \end{cases}
\]
\end{definition}

\begin{definition} \ \\
    Тип (монотип) "--- выражение в грамматике $ \begin{bnf} \tau ::= \alpha | \tau \rightarrow \tau | (\tau) \end{bnf} $. \\
    Типовая схема (политип) "--- выражение в грамматике $ \begin{bnf} \sigma ::= \tau | \forall \alpha . \sigma \end{bnf} $.
\end{definition}

\begin{statement}
    $\rank(\tau) = 0$, $\rank(\sigma) = 1$.
\end{statement}

\begin{example} Ранг экзистенциального типа
\begin{align*}
    \rank(\exists \alpha . \beta) &= \rank(\forall \gamma . (\forall \alpha . \beta \rightarrow \gamma) \rightarrow \gamma) \\
    &= \max(\rank((\forall \alpha . \beta \rightarrow \gamma) \rightarrow \gamma), 1) \\
    &= \max(\max(\rank(\forall \alpha . \beta \rightarrow \gamma) + 1, \rank(\gamma)), 1) \\
    &= \max(\max(2, 0), 1) = 2
\end{align*}
\end{example}

\begin{definition}
    $\sigma_1$ "--- подтип $\sigma_2$, если существует подстановка
            $[\alpha_1 \coloneqq \theta_1, \alpha_2 \coloneqq \theta_2 \ldots \alpha_n \coloneqq \theta_n]$:
    \begin{enumerate}
        \item $\sigma_1 = \forall \beta_1 \ldots \forall \beta_k . \tau [\alpha_1 \coloneqq \theta_1 \ldots \alpha_n := \theta_n]$,
            $\alpha_i$ не входит свободно в $\theta_j$.
        \item $\sigma_2 = \forall \alpha_1 \ldots \forall \alpha_n \tau$
    \end{enumerate}
\end{definition}

\begin{definition}[Правила вывода в системе Хиндли-Милнера]
\begin{gather*}
    \infer[\text{(Аксиома)}]{\Gamma, x : \sigma \vdash x : \sigma}{} \qquad
    \infer[\text{(Уточнение), $\sigma_1$ "--- подтип $\sigma$}]{\Gamma \vdash e : \sigma'}{\Gamma \vdash e : \sigma} \\
    \infer[\text{(Обобщение), } \sigma\notin\FV(\Gamma)]{\Gamma \vdash e : \forall \alpha . \sigma}{\Gamma \vdash e : \sigma} \qquad
    \infer[\text{(Абстракция)}]{\Gamma \vdash \lambda x . e : \tau' \rightarrow \tau}{\Gamma, x : \tau' \vdash e : \tau} \\
    \infer[\text{(Подстановка, применение)}]
        {\Gamma \vdash e e' : \tau}{\Gamma \vdash e : \tau' \rightarrow \tau && \Gamma \vdash e' : \tau'} \qquad
    \infer[\text{(Let)}]
        {\Gamma \vdash \lett{x=e}{e'=\tau}}
        {\Gamma \vdash e : \sigma && \Gamma \setminus \set{x}, x : \sigma \vdash e' : \tau}
\end{gather*}
\end{definition}
\todo попросить у Д.Г. оригинальную статью.

Грамматика выражения выглядит следующим образом:
\[
\begin{bnf}
    \Lambda ::= x | \lambda x . \Lambda | \Lambda \Lambda | (\Lambda) | \lett{x=\Lambda}{\Lambda}
\end{bnf}
\]

\subsection{\texorpdfstring{Алгоритм W}{Algorithm W}}
\begin{statement}
    Задача вывода типа в Х-М разрешима.
\end{statement}
Обозначения: $\overline{\Gamma(\tau)} = \forall \alpha_1 \ldots \forall \alpha_n . \tau$, где $\alpha_1 \ldots \alpha_n \notin \FV(\Gamma)$
(замыкание всех несвязанных переменных); $\Gamma_x = \Gamma \setminus \set{x}$.

Определим $W(\Gamma, e) = (S, \tau)$, принимающую контекст и выражение, и возвращающую такую подстановку и тип,
что $S(\Gamma) \vdash e : \tau$.
\begin{center}
\begin{tabular}{Sl Sl Sl} \hline
    $e \equiv x$
        & $x : \forall \alpha_1 \ldots \alpha_k . \tau' \in \Gamma$
        & $S'=\mathrm{Id}$ \\ \hline
    $e \equiv e_1 e_2$
        & \begin{tabular}[t]{@{}Sl@{}}
            $W(\Gamma, e_2) = (S_2, \tau_2)$ \\
            $W(S_2(\Gamma), e_1) = (S_1, \tau_1)$ \\
            $U(S_1(\tau_1), \tau_2 \rightarrow \beta) = V$, где $\beta$ "--- новый тип
        \end{tabular}
        & $(V(S_1 \circ S_2), V(\beta))$ \\ \hline
    $e\equiv\lambda x . e$
        & $W(\Gamma_x \cup \set{x : \beta}, e) = (S_1, \tau_1)$, $\beta$ "--- новый тип
        & $(S_1, S_1(\beta \rightarrow \tau_1))$. \\ \hline
    $e\equiv\lett{x=e_1}{e_2}$
        & \begin{tabular}[t]{@{}Sl@{}}
            $W(\Gamma, e_1) = (S_1, \tau_1)$ \\
            $W(S_1(\Gamma_x \cup \{x : S_1(\overline{\Gamma(\tau_1)})\}), e_2) = (S_2, \tau_2)$
        \end{tabular}
        & $(S_2 \circ S_1, \tau_2)$ \\ \hline
\end{tabular}
\end{center}
\todo примеры
